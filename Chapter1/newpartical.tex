\section{新粒子の探索}
2012年のLarge Hadron Collider (LHC)でのヒッグス粒子の発見により、素粒子標準理論で予想される全ての素粒子が見つかった。
しかし、初期宇宙では物質と同量の反物質が存在していたはずだが、現在では反物質のほとんど存在しないことや、
宇宙の構成要素の約95%を占める、ダークマターとダークエネルギーの正体、重力を媒介する素粒子についてなど、
標準理論では説明できない事象はさまざま存在する。
そのため、素粒子物理学では、標準理論を超える新しい物理の発見・解明のための研究が進められており、このような事象を解明することが素粒子物理学の研究目的の一つである。

%銀河の回転速度が中心からの距離によらずに一定であることや、重力レンズ効果からダークマターの質量も測定されているため、
%ダークマターやダークエネルギーの存在は確実視されているが、その正体について発見や解明はされていない。

このような標準理論を超える未知の粒子の存在を示唆する理論の1つとして、超対称性理論がある。
この理論によると、既知の素粒子それぞれに、ボーズ粒子とフェルミ粒子の特徴を入れ替えた超対称性粒子が存在するとされており、
この中にダークマターに該当する粒子があるのではないかと考えられている。
これらの粒子は質量が大きいとされており、さまざまな高エネルギーの加速器実験で探索されている。

