加速器の高輝度化に向けた内部飛跡検出器として、高い位置分解能と時間分解能を併せ持つAC-LGAD検出器の開発を行っている。
本研究では、浜松ホトニクス社と共同で試作した増幅層がなく増幅機構を持たないP-Intrinsic-N (PIN)ダイオードと、
LGAD検出器を作成し、増幅率と時間分解能についての評価を行った。

まず、赤外線パルスレーザーを用いて、LGADとPINの信号の大きさの電圧依存性を測定した。
LGADの信号の大きさは、増幅層によって意図的に作り出された高電場によって、指数関数的に増加することがわかった。
PINの信号の大きさは、電圧上昇によって空乏層が広がることで、信号の大きさが大きくなることがわかった。
LGADとPINの信号の大きさの比を取ることで、LGAD検出器の増幅率の電圧依存性を求めた。
印加電圧が上昇すると、増幅率が指数関数的に上昇することがわかった。
これは、LGADの信号の大きさの増加の影響がPINの信号の大きさに比べて、非常に大きいためであると考える。

LGADとPINの時間分解能の電圧依存性の比較を行った結果、
LGADは電圧が上昇すると、時間分解能が最小値をとり、それ以上の電圧を印加すると時間分解能が上昇する様子が見られた。
PINは空乏化が生じている電圧では、電圧上昇に伴って時間分解能も小さくなったが、空乏化が終わると、時間分解能の変化がなくなる様子が見られた。
LGADとPINの時間分解能の比較から、検出器に増幅効果があることで時間分解能が5 倍程度に改善することがわかった。
増幅率と時間分解能の測定結果から、増幅率がおよそ20〜35 倍のときに、最も良い時間分解能およそ10 psを得られることがわかった。

立ち上がり時間$t_{\rm{r}}$、60%から40%の波高の差$S$、ノイズ$\sigma_{\rm{n}}$を測定し、式\ref{eq_Jitter_2} からジッター$\sigma_{\rm{j}}$を計算から求めた。
赤外線パルスレーザーによる測定では、タイムウォーク$\sigma_{\rm{tw}}$とランダウノイズ$\sigma_{\rm{L}}$の影響が少ない状況で時間分解能$\sigma_{\rm{t}}$を測定することができる。
そのため、レーザー測定から求めた時間分解能$\sigma_{\rm{t}}$と、計算から求めたジッター$\sigma_{\rm{j}}$との比較から、時間分解能が悪化する原因について調べた。

LGADの立ち上がり時間は、電圧上昇により小さくなることがわかった。これは、電圧上昇による電場の増大によって、電子正孔対の速度が大きくなったからであると考える。
また、運転電圧を超えると、立ち上がり時間が増加してしまった。これは、電子正孔対が飽和ドリフト速度に達したことに加え、電圧上昇によるノイズが増えたためであると考える。

LGADのノイズは約4 mVで、PINのノイズは約2.6 mVで電圧増加に対して変化がなく、ほぼ一定であった。LGADのノイズはPINに比べて約1.6倍大きい結果となった。
電子雪崩の影響が大きくなる200 Vでは、LGADのノイズがおよそ4.7 mVに上昇することがわかった。

これまでの測定結果から、ジッターの増幅率依存性を調べると、増幅率が大きくなるほど、ジッターは減少することがわかった。
これは、立ち上がり時間が早くなること加えて、信号の大きさが増加する影響が非常に大きいからであると考える。
また、200 Vでは電子雪崩によるノイズの増加によって、ジッターは少しだけ増加することがわかった。
レーザー測定から求めた時間分解能と、計算から求めたジッターとの比較から、ジッターのみでは説明できない時間分解能の悪化が見られることがわかった。

レーザー測定から求めた時間分解能から、ジッターとレーザーのタイミングジッターの影響を差し引いた結果、増幅率が大きくなることで増加する過剰ノイズと思われる影響がみられた。
過剰ノイズは、増幅率が35 倍以上になると、ジッターと比べて支配的になることがわかった。
以上の結果と第3章の時間分解能が良い増幅率が20〜35倍という結果から、時間分解能が良い増幅率では過剰ノイズの影響は小さく、時間分解能にほとんど影響しないと考えられる。

%時間分解能が良い波形と悪い波形を比較することにより、悪化の原因を調べた。
%時間分解能が悪化した時の波形は、波高の揺らぎが大きくなっており、波高が大きくなるほど最大波高の時間が遅いことがわかった。



%そのため、時間分解能が増加する要因として、信号由来の波高の揺らぎ$\sigma_{\rm{s}}$による効果があるのではないかと考え、その効果の評価を進めた。
%この効果を、式\ref{eq:Multiplication_Noise} を仮定して評価したところ、
%%増幅率が上昇することで、信号由来の波高の揺らぎによる効果が大きくなる様子が見られた。
%増幅率がおよそ15 倍以上になると、増幅率に比例して波高の揺らぎによる効果が大きくなる様子が見られた。
%LGAD検出器の時間分解能$\sigma_{\rm{t}}$が上昇してしまう原因は、検出器由来の波高の揺らぎによる効果の影響であると考える。

%以上の結果から、AC-LGAD検出器の時間分解能は、タイムウォーク$\sigma_{\rm{tw}}$、
%ジッター$\sigma_{\rm{j}}$ 、ランダウノイズ$\sigma_{\rm{L}}$ に加わる要因があり、特に増幅率が大きい時にその効果が顕著であることがわかった。


