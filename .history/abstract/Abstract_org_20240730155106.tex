\begin{center}
  {\Large 卒業論文 2023年度(令和5年度)}

  {\LARGE 高い時間分解能を持つAC-LGAD検出器の\\増幅率およびああ時間分解能の研究}
\end{center}

\quad

\begin{flushright}
    202012130 堀越 一生

    指導教員 廣瀬 茂輝
\end{flushright}

\quad

{\large \gt 論文要旨}

%\begin{comment}
  標準理論を超える物理現象や新粒子を発見するために、加速器実験は高輝度化と高エネルギー化を繰り返しながら発展してきた。
  高輝度化によって1バンチあたりの衝突数が増加すると、衝突で生成される粒子数が増える。
  生成された荷電粒子の飛跡を測定するために使われるシリコン製の荷電飛跡検出器は、電極サイズを小さくすることで高い位置分解能を実現する。
  それに加えて、検出器に高い時間分解能を持つことで粒子の飛跡に時間情報を追加することができる。
  それによって、高輝度化に伴うパイルアップによって困難とされる、衝突点と飛跡の結びつけが可能になる。
  
  Low-Gain-Avalanche-Diode(LGAD)検出器は、$p$型シリコン半導体をベースにしたシリコン半導体検出器で、
  $pn$接合を作るために形成された表面の$n^+$層の直下に、増幅層としてアクセプター濃度が高い$p^+$層をドープした構造になっている。
  検出器に逆バイアス電圧をかけると、高濃度の$p^+$層によって、局所的に高電場領域を作り出すことができる。
  検出器内で生成された電子正孔対が、高電場による電子雪崩によって増幅されることで、立ち上がりが速くタイミングが揃った信号を出力することができ、
  そのため良い時間分解能で荷電粒子を検出できる。
  中でもAC-LGAD検出器は、増幅層を読み出し電極ごとに分割しないで一様に形成し、酸化膜を介して電極に誘起された信号を検出する。
  このような構造から、隣接チャンネルへの信号のクロストークが問題であったが、
  不純物濃度と酸化膜の厚さを最適化することでクロストークを抑制したAC-LGAD検出器では、30 psの時間分解能が確認されている。
  
  %LGAD検出器は増幅した信号を出力できるため、Application-Specific-Integrated-Circuit(ASIC)内のアンプの増幅率を小さく設定することができ、
  %ASICの消費電力の削減や発熱を抑えることが可能になる。
  %そのため、LGAD検出器の増幅率を求めることで、ASIC 内のアンプの開発に貢献することができる。
  
  AC-LGAD検出器の時間分解能は、信号の大きさの違いによるタイムウォーク$\sigma_{\rm{tw}}$の影響、回路にのるノイズによる影響(ジッター$\sigma_{\rm{j}}$)、
  荷電粒子がセンサー内に落とすエネルギーの非一様性の影響(ランダウノイズ$\sigma_{\rm{L}}$)の3つが大きく寄与する。
  タイムウォークは信号波高に対して50%の位置に閾値を設定する(constant fraction方式)を使用することで、その影響を抑制できる。
  
  本研究では、AC-LGAD 検出器の ASIC の開発に向けた基礎特性の理解として、最も良い時間分解能を実現する増幅率を決定することと、
  今後のAC-LGAD検出器の時間分解能の向上をはかるために、特に増幅率が高い場合に時間分解能が悪化する原因について理解することを目的とする。
  
  まず最初に、%AC-LGAD検出器に対する逆電圧を変えながら増幅率と時間分解能の関係を測定し、最も良い時間分解能を実現する増幅率を調べるために、
  LGAD検出器と同じ構造を持つ、増幅層がなく増幅機構を持たないP-Intrinsic-N (PIN)ダイオードを作成した。%増幅層があるAvalanche-Photo-Diode (APD)を作成した。
  これらのサンプルに対し、同じ強度の赤外線パルスレーザーを入射して、信号の大きさと時間分解能を測定する。
  LGAD検出器とPINの信号の大きさの比を取ることで、LGAD検出器の増幅率が求められる。
  このような測定から増幅率がおよそ20〜35 倍のときに、最も良い時間分解能およそ10 psを得られることがわかった。
  
  赤外線パルスレーザーの場合は、センサー内で一様に電子正孔対を生成するため、
  タイムウォークおよびランダウノイズの影響が少ない状況で時間分解能を測定できる。
  また、ノイズを閾値付近の微小時間あたりの波高の変化量で割ることによって、ジッターを求めることができる。
  そのため、ノイズに加えて、微小時間あたりの波高の変化量を求めるために、信号の立ち上がり時間$t_{\rm{r}}$、信号の大きさ$S$の測定をしてジッターを求めた。
  レーザー測定から求めた時間分解能と、計算から求めたジッターとの比較から、ジッターのみでは説明できない時間分解能の悪化が見られる
  ことがわかった。
  
  以上の結果から、AC-LGAD検出器の時間分解能は、タイムウォーク$\sigma_{\rm{tw}}$、
  ジッター$\sigma_{\rm{j}}$ 、ランダウノイズ$\sigma_{\rm{L}}$ に加わる要因があり、特に増幅率が大きい時にその効果が顕著であることを明らかにした。
  その要因について理解を深めることが、今後のAC-LGAD検出器のさらなる時間分解能向上の鍵となる。
  
  
  %そのため、時間分解能を悪化させる要素は他にあると考え、時間分解能が悪化する前後の波形を比較すると、波高の揺らぎが大きくなることを確認した。
  %検出器の信号由来の影響の波高の揺らぎを$\sigma_{s}$とした。波高の揺らぎからノイズを引いた値を微小時間あたりの波高の変化量で割ることによって、時間分解能への影響を求めた。
  %その結果、増幅率に比例してこの影響が大きくなることがわかった。
  %AC-LGAD検出器の時間分解能と計算から求めたジッター、波高の揺らぎによる影響の増幅率依存性を比較することで、
  %増幅率が大きくなると、ジッターは小さくなるが、波高の揺らぎによる影響が大きくなり、それが時間分解能の悪化に寄与することがわかった。
  %以上の結果から、AC-LGAD検出器の時間分解能が悪化する影響する原因が、波高の揺らぎによる影響であり、
  %AC-LGAD検出器の時間分解能は、タイムウォーク$\sigma_{tw}$、ジッター$\sigma_j$、ランダウノイズ$\sigma_L$の3つに加えて、
  %新たに、検出器の信号由来の影響が時間分解能に寄与することを主張する。








%\end{comment}
