\chapter{謝辞}

本研究および卒業論文の執筆にあたって、多くの方々の力添えをいただいたことをこの場で感謝申し上げます。

指導教員である廣瀬茂輝先生は、卒業論文の添削や書き方のアドバイスを始め、口頭発表や資料作成についてなど大変多くのご指導をいただきました、
さらに、一人前の研究者になるために必要な考え方や、それに向けてのアドバイス等も含めて本当に手厚いサポートしていただきました。

高エネルギー加速器研究機構の中村浩二さんには、測定装置や検出器の仕組みや使い方、苦手だったデータの解析方法について、お忙しい中でも丁寧に教えていただきました。
そして、研究結果について一緒に議論してくださった際には、自分では考えつかないような目線からの意見やアドバイスをいただき、新たな見方や考え方を得ることができました。

原和彦先生には、R&Dミーティングや卒業論文の中間発表の際に、LGAD検出器の性質や偏向板の原理をはじめ、様々な知識や研究のアドバイスをいただきました。
教えていただいた知識やアドバイスを生かして、測定結果や研究についての新たな考えや結果を出すことができました。

LGAD研究グループの先輩の北彩友海さん、今村友香さん、西野純矢さん、同期の村山由亞くんには、研究室に入ってわからないことだらけだった自分に、
測定装置の使い方や解析方法、LGAD検出器の仕組みなどを丁寧に教えていただきました。。一緒に議論を行い、様々な意見をもらうことで私の研究をより深く進めることができました。

素粒子実験研究室の皆さんにはゼミやセミナーなどを通して、数多くの素粒子物理学の基礎知識や研究内容について学ぶことができました。
また、LaTexの使い方について教えていただき卒業論文の執筆をスムーズに進めることができました。

卒業研究を通して、たくさんの知識と経験、考え方を身につけることができました。
この1年間を乗り越えられたのは、皆さんのご支援があったからこそです。
ここに卒業研究を支援していただいた皆さんへ感謝の意を表します。


\begin{flushright}
    堀越一生
\end{flushright}